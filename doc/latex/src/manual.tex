\documentclass[11pt]{article}
\usepackage{CJK}
\usepackage{verbatim}
\usepackage{graphicx}
\usepackage{color}
\usepackage{fancybox}
\usepackage{alltt}
%\usepackage{indentfirst}

% user-defined
\newcounter{yypagecounter}[section]
\newcommand{\yynewpage}[1] {
\stepcounter{yypagecounter}
\ovalbox{
  \begin{minipage}{12cm}
  \bf{\arabic{yypagecounter} #1}
  \end{minipage}
}\newline
\newline
}

\newcommand{\shellcmd}[1] {
\fbox{
\begin{minipage}{12cm}
\textcolor{black}{#1}
\end{minipage}}
\newline
}

\begin{CJK}{UTF8}{song}
\title{YFS集群文件系统配置手册}
\author{For yfs-1.9.4}
\date{\today}
\end{CJK}

\begin{document}
\begin{CJK}{UTF8}{song}

\maketitle
\tableofcontents 

\newpage
\section{系统需求}
\begin{itemize}
  \item 支持平台
    \begin{itemize}
      \item GNU/Linux是本产品开发和运行的平台,其它平台暂不予支持
    \end{itemize}
  \item 软件需求
    \begin{itemize}
      \item libc运行时库,更新到最新稳定版本。
      \item ssh必须安装并且保证sshd一直运行,以便用YFS脚本管理远端守护进程。
      \item Python2.4以上版本。
      \item 如果需要开启web监控,还需要安装webpy。
    \end{itemize}
\end{itemize}

\section{安装软件}
\noindent 如果你的集群尚未安装所需软件,你得首先安装它们。

\noindent 以Ubuntu Linux为例:\\
\shellcmd{\$ sudo apt-get install openssh-server openssh-client}

\noindent Centos linux:\\
\shellcmd{\$ yum install openssh-server openssh-client}

\section{运行集群准备工作}
\subsection{安装}
\noindent 创建/sysy/yfs目录,该目录是YFS集群文件系统所需的工作目录\\
\shellcmd{\$ mkdir -p /sysy/yfs}

\noindent 拷贝yfs.tar.gz到该目录下并解压\\
\shellcmd{
\$ cp yfs.tar.gz /sysy/yfs\\
\$ cd /sysy/yfs\\
\$ tar -zxvf yfs.tar.gz}

\subsection{配置集群}

\begin{itemize}
  \item master.cfg里包含了mds,镜像mds,ftp,nfs服务器的ip或主机名配置,
    按需求将相关主机名和ip地址填入对应的项,
    集群在启动时会在对应的服务器上启动对应的服务。
    下面是一个例子:\\
\shadowbox{
\begin{minipage}{10cm}
\begin{verbatim}
  # MDS服务器地址
  host=192.168.1.1

  # 镜像MDS服务器地址
  [secondarymds]
  host=192.168.1.2

  # FTP服务器地址
  [yftp]
  host=192.168.1.3

  # NFS服务器地址
  [ynfs]
  host=192.168.1.3
\end{verbatim}
\end{minipage}
}

  \item slaves.cfg里面包含了cds服务器所在的服务器ip或主机名,
    其中number这个参数知道了在具体某一台服务器上将要启动的cds个数,
    需要注意的是,\textcolor{red}{cds必须从1开始连续增长,
    如cds1,cds2,cds3\ldots,而不能是cds1,cds2,cds4,cds7\ldots},
    下面是一个例子:\\
\shadowbox{
\begin{minipage}{10cm}
\begin{verbatim}
  [cds1]
  host=192.168.1.5
  number=1
  [cds2]
  host=192.168.1.6
  number=2
  [cds3]
  host=192.168.1.7
  number=3
\end{verbatim}
\end{minipage}
}
\end{itemize}

\section{部署YFS到集群所有服务器}

\section{运行}
\shellcmd{
\$ /sysy/yfs/app/sbin/yfs-daemon.py -s yfs\dotfill启动yfs文件系统\\
\$ /sysy/yfs/app/sbin/yfs-daemon.py -s all\dotfill启动yfs, yftp, ynfs}

注意:单独启动YFTP或者YNFS不会启动YFS,如果你需要启动YNFS,则需要\\
\shellcmd{
\$ /sysy/yfs/app/sbin/yfs-daemon.py -s yfs\\
\$ /sysy/yfs/app/sbin/yfs-daemon.py -s ynfs}

\end{CJK}
\end{document}
